\documentclass[letterpaper,
thesis]{rackham}

\title{Untitled dissertation template}
    \author{Jane R. Doe}
    \authoremail{jdoe@umich.edu}
    \orcidid{9999-9999-9999-9999}

% Department
\department{Mathematics}

% Year of completion
\year=2022

\dedication{This manual is dedicated to all doctoral students at the
University of Michigan's Horace H. Rackham School of Graduate Studies.

Also to my plants.}

\acknowledgments{This template has been modified by a lot of people over
the years. To the best of my knowledge, it's original version was
written by Jin Ji in 1988. Modifications to it have since been made by
Roque D.~Oliveira in 1992, Jason Gilbert in 2008, Derek Dalle in 2011,
and Umang Varma in 2019. It is possible that other people have modified
this document (in fact, there is a unattributed change in the changelog
dated 1989.11.29).}

\preface{This disseration is a sample document using the
\texttt{rackham.cls} template.}
\committee{Professor John D. Brown, Co-Chair \\
 Professor Emeritus Ann A. Smith, Co-Chair}


% Commands to hide or show lists of figures, tables, etc.
%\showlistoffigures
%\showlistoftables
%\showlistofmaps
%\showlistofillustrations
%\showlistofappendices
%\showlistofabbreviations
%\showlistofacronyms
%\showlistofsymbos
% OR
%\hidelistoffigures
%\hidelistoftables
%\hidelistofmaps
%\hidelistofillustrations
%\hidelistofappendices
%\hidelistofabbreviations
%\hidelistofacronyms
%\hidelistofsymbols

% Definition of any abbreviations used.
%\abbreviations{
% \acro{LS&A}{Literature Science and Arts}
%}

% Some abstract text
\abstract{With walk through basic steps of using this \LaTeX~template.}

\titleformat{\section} %\titleformat*{\section}{\center\bfseries}
    {\center\normalfont\Large\bfseries}{\thesection}{1em}{} 
\titleformat{\subsection} %\titleformat*{\subsection}{\bfseries}
    {\normalfont\large\bfseries}{\thesubsection}{1em}{}
\titleformat{\subsubsection} %\titleformat*{\subsubsection}{\bfseries}
    {\normalfont\normalsize\bfseries}{\thesubsubsection}{1em}{}
\begin{document}

\bookmarksetup{startatroot}

\hypertarget{introduction}{%
\chapter{Introduction}\label{introduction}}

\emph{TODO} Create an example file that demonstrates the formatting and
features of your format.

You can learn more about controlling the appearance of PDF output here:
\url{https://quarto.org/docs/output-formats/pdf-basics.html}

\bookmarksetup{startatroot}

\hypertarget{using-this-template}{%
\chapter{Using This Template}\label{using-this-template}}

The approach to this template is to result in
\LaTeX\textasciitilde source files (i.e., \texttt{.tex} files) that are
as simple as possible \%(Knuth 1986).\\
This is particularly useful for the first few pages, for example the
title page, dedication, and abstract page, which are difficult to make
in \LaTeX\textasciitilde and are supposed to go in a certain order.

You are welcome to modify \texttt{thesis-umich.cls} to suit your needs
or keep up with modifications to Rackham guidelines. I used this
template to submit my dissertation in 2019, but Rackham can change their
rules at any time. If you have not modified a \texttt{.cls} file before,
but know how to define commands in \TeX, you should not have much
trouble, apart from the sneaky \verb=@= character, which behaves like a
letter in \texttt{.cls} files but not in \texttt{.tex} (see
\href{https://tex.stackexchange.com/q/8351/21027}{``What do
  \texttt{\textbackslash makeatletter} and \texttt{\textbackslash makeatother}
do?'' on \TeX\ StackOverflow} and
\href{https://tug.org/pipermail/tugindia/2002-January/000178.html}{``Makeatletter
and Makeatother'' on the \TeX\ Users Group} for more information).

Finally, you should be able to use your preferred bibliography manager
(BibTex, BibLaTeX, NatBib, etc.). If the
\texttt{\textbackslash bibliography} command is giving you trouble, look
in \texttt{thesis-umich.cls}, because it was modified there. You can
also look at that code for help with formatting bibliographies that are
displayed with other commands.

\bookmarksetup{startatroot}

\hypertarget{discussion}{%
\chapter{Discussion}\label{discussion}}

discuss results

\hypertarget{refs}{}
\begin{CSLReferences}{1}{0}
\leavevmode\vadjust pre{\hypertarget{ref-knuth_1986_tex}{}}%
Knuth, Donald E. 1986. \emph{The TeXbook}. Addison-Wesley Professional.

\end{CSLReferences}


\appendix
\chapter{Example Appendix}

\section{Lists Including the Appendices}
The command
\begin{verbatim}
\showlistofappendices
\end{verbatim}
must appear in the preamble if there are more than one appendices.  For
some reason, Rackham does not want the individual appendices and their
sections to appear in the Table of Contents, so a special List of
Appendices page (which must occur in the Table of Contents!) is required
as a sort of extension to the Table of Contents.


%\bibliographystyle{alpha}
%\bibliography{references.bib}

\end{document}
